\documentclass{Resources/UoBLab1}
\pubyear{2025}
\subjectarea{BioRobotics}
\usepackage{hyperref}
\hypersetup{colorlinks=true, linkcolor=blue, urlcolor=cyan}
\usepackage[ruled, linesnumbered, vlined, slovak]{algorithm2e}
\renewcommand{\algorithmcfname}{ALGORITHM}
\SetAlFnt{\small}
\SetAlCapFnt{\small}
\SetAlCapNameFnt{\small}
\SetAlCapHSkip{0pt}
\IncMargin{-\parindent}

\usepackage{tcolorbox}
\tcbuselibrary{minted,breakable,xparse,skins}

\definecolor{bg}{gray}{0.95}
\DeclareTCBListing{mintedbox}{O{}m!O{}}{%
  breakable=true,
  listing engine=minted,
  listing only,
  minted language=#2,
  minted style=default,
  minted options={%
    linenos,
    gobble=0,
    breaklines=true,
    breakafter=,,
    fontsize=\small,
    numbersep=8pt,
    #1},
  boxsep=0pt,
  left skip=0pt,
  right skip=0pt,
  left=25pt,
  right=0pt,
  top=3pt,
  bottom=3pt,
  arc=5pt,
  leftrule=0pt,
  rightrule=0pt,
  bottomrule=2pt,
  toprule=2pt,
  colback=bg,
  colframe=orange!70,
  enhanced,
  overlay={%
    \begin{tcbclipinterior}
    \fill[orange!20!white] (frame.south west) rectangle ([xshift=20pt]frame.north west);
    \end{tcbclipinterior}},
  #3}

\begin{document}
\firstpage{1}

\title{Lab 1: Introduction to LSL and EMG Signals}

\maketitle

\begin{abstract}
This lab introduces the Lab Streaming Layer (LSL) protocol for biosignal acquisition and analysis of electromyogram (EMG) signals. Students will collect EMG data from the Myo Armband and BioRadio 150, visualize signals in real-time, apply signal processing techniques, and explore proportional control concepts used in myoelectric prosthetics.
\\[0.5em]
\textbf{Please read through the entire document before completing the lab.}
\end{abstract}

\section{Learning Objectives}

By the end of this lab, you will be able to:
\begin{enumerate}
    \item Explain the Lab Streaming Layer (LSL) architecture and its role in biosignal acquisition
    \item Set up and operate the Myo Armband and BioRadio 150 for EMG recording
    \item Use Python tools to discover, visualize, and record LSL streams
    \item Apply signal processing techniques: filtering, rectification, and envelope extraction
    \item Extract time and frequency domain features from EMG signals
    \item Demonstrate proportional control using EMG signals
\end{enumerate}

\section{Lab Materials}

\subsection{Files to Edit}
\begin{enumerate}
    \item \textit{notebooks/Lab1\_EMG\_Analysis.ipynb}: Jupyter notebook for EMG analysis
    \item Data files you collect during the lab
\end{enumerate}

\subsection{Reference Files (Do Not Modify)}
\begin{itemize}
    \item \texttt{src/lsl\_utils.py}: LSL stream discovery and recording utilities
    \item \texttt{src/visualizer.py}: Real-time EMG visualization tool
    \item \texttt{src/emg\_processing.py}: Signal processing functions
    \item \texttt{src/proportional\_control.py}: Proportional control demonstration
\end{itemize}

\section{Background: Lab Streaming Layer (LSL)}

\subsection{What is LSL?}

The \textbf{Lab Streaming Layer} is a system for unified collection of time-series data in research experiments. It provides:

\begin{itemize}
    \item \textbf{Network transparency}: Data streams over the local network
    \item \textbf{Time synchronization}: All streams share a common clock
    \item \textbf{Device independence}: Works with many different sensors
    \item \textbf{Language support}: APIs for Python, C++, MATLAB, and more
\end{itemize}

\subsection{LSL Architecture}

\begin{verbatim}
    ┌──────────────┐    ┌──────────────┐
    │   BioRadio   │    │     Myo      │
    │   (Device)   │    │  (Armband)   │
    └──────┬───────┘    └──────┬───────┘
           │                   │
           ▼                   ▼
    ┌─────────────────────────────────────┐
    │    Lab Streaming Layer (Network)    │
    └─────────────────────────────────────┘
           │                   │
           ▼                   ▼
    ┌──────────────┐    ┌──────────────┐
    │  Visualizer  │    │   Recorder   │
    └──────────────┘    └──────────────┘
\end{verbatim}

\textbf{Key Concepts:}
\begin{itemize}
    \item \textbf{Stream}: A continuous source of time-series data
    \item \textbf{Outlet}: Publishes data to the LSL network
    \item \textbf{Inlet}: Receives and consumes data from a stream
    \item \textbf{XDF}: File format for storing LSL recordings
\end{itemize}

\section{Environment Setup}

\subsection{Installing the Conda Environment}

This lab uses a Conda environment to manage dependencies. Conda does not require administrator privileges.

\begin{enumerate}
    \item Open a terminal (Command Prompt on Windows, Terminal on macOS/Linux)
    
    \item Navigate to the lab directory:
    \begin{minted}{bash}
    cd path/to/Lab_1
    \end{minted}
    
    \item Create the environment:
    \begin{minted}{bash}
    conda env create -f environment.yml
    \end{minted}
    
    \item Activate the environment:
    \begin{minted}{bash}
    conda activate biorobotics
    \end{minted}
    
    \item Verify the installation:
    \begin{minted}{bash}
    python -c "import pylsl; print('LSL version:', pylsl.__version__)"
    \end{minted}
\end{enumerate}

\textbf{Note:} You must activate the \texttt{biorobotics} environment each time you open a new terminal.

\section{Device Setup: Myo Armband}

The Myo Armband provides 8 channels of surface EMG (sEMG) at 200 Hz, plus IMU data.

\subsection{Hardware Setup}

\begin{enumerate}
    \item Plug the Myo Bluetooth dongle into the computer
    \item Charge the Myo if needed (LED will pulse when charging)
    \item Put on the armband:
    \begin{itemize}
        \item Position the Thalmic logo on the outer forearm (extensor side)
        \item The LED/status bar should point toward your hand
        \item Adjust the clips for a snug fit
    \end{itemize}
    \item Turn on the Myo by pressing and holding the button until it vibrates
\end{enumerate}

\subsection{Software Setup (Windows)}

\begin{enumerate}
    \item Open \textbf{MyoConnect} from the Start menu
    \item Wait for the Myo to connect (icon appears in system tray)
    \item Perform the \textbf{sync gesture}: wave your hand outward
    \item You should feel a vibration confirming sync
\end{enumerate}

\subsection{Streaming to LSL}

Once MyoConnect is running and synced, start the LSL streamer:

\begin{minted}{bash}
python src/myo_interface.py
\end{minted}

This creates two LSL streams:
\begin{itemize}
    \item \texttt{Myo\_EMG}: 8-channel EMG at 200 Hz
    \item \texttt{Myo\_IMU}: Orientation, acceleration, and gyroscope data
\end{itemize}

\section{Device Setup: BioRadio 150}

The BioRadio 150 is a research-grade biosignal amplifier with configurable channels.

\subsection{Electrode Placement for Forearm EMG}

\begin{enumerate}
    \item Clean the skin with alcohol wipes
    \item Place electrodes as follows:
    \begin{itemize}
        \item \textbf{Channel 1}: Inner forearm (flexor muscles)
        \item \textbf{Channel 2}: Outer forearm (extensor muscles)
        \item \textbf{Ground}: Elbow (bony area with minimal muscle)
    \end{itemize}
    \item Connect snap leads to the electrodes
    \item Connect leads to the BioRadio harness
\end{enumerate}

\subsection{BioCapture Configuration}

\begin{enumerate}
    \item Plug in the USB receiver
    \item Open BioCapture software
    \item Click \textbf{Connect} from the toolbar
    \item Configure channels:
    \begin{itemize}
        \item Select \textbf{Differential} mode
        \item Set sampling rate to 960 Hz or higher
        \item Enable desired channels
        \item Select EMG for channel type
    \end{itemize}
    \item Click \textbf{Program Device}
    \item Click the green triangle to start streaming
\end{enumerate}

\textbf{Note:} Verify sampling rate is at least 2× the highest frequency of interest (Nyquist theorem).

\section{Using the Visualization Tools}

\subsection{Real-time EMG Visualizer}

The visualizer lets you monitor EMG signals in real-time:

\begin{minted}{bash}
python src/visualizer.py
\end{minted}

\begin{enumerate}
    \item Click \textbf{Refresh} to scan for available LSL streams
    \item Select your device from the dropdown
    \item Click \textbf{Connect} to start visualization
    \item Adjust the time window as needed
    \item Click \textbf{Record} to start recording data
    \item Click \textbf{Save} to export to CSV
\end{enumerate}

\subsection{Testing Without Hardware}

You can test the tools using mock (simulated) data:

\begin{minted}{bash}
python src/myo_interface.py --mock
\end{minted}

Then open the visualizer and connect to the mock stream.

\section{Data Collection Task}

\subsection{Gesture Recording Protocol}

Record the following gestures with both devices:

\begin{enumerate}
    \item \textbf{Rest}: Relaxed hand, no muscle activation (5 seconds)
    \item \textbf{Fist}: Closed fist, moderate grip (3 seconds)
    \item \textbf{Open}: Fingers extended and spread (3 seconds)
    \item \textbf{Wrist Flexion}: Bend wrist toward palm (3 seconds)
    \item \textbf{Wrist Extension}: Bend wrist away from palm (3 seconds)
\end{enumerate}

\textbf{Recording Procedure:}
\begin{enumerate}
    \item Start the visualizer and verify good signal quality
    \item Begin recording
    \item Announce each gesture verbally (for reference)
    \item Perform 5 repetitions of each gesture
    \item Stop recording and save the file
    \item Note the timestamps for each gesture
\end{enumerate}

\subsection{File Naming Convention}

Use descriptive file names:
\begin{verbatim}
{participant}_{device}_{session}.csv

Examples:
john_myo_session1.csv
john_bioradio_session1.csv
\end{verbatim}

\section{Proportional Control Demo}

This demo shows how EMG amplitude can control external systems---the basis of myoelectric prosthetics.

\begin{minted}{bash}
python src/proportional_control.py
\end{minted}

\subsection{Instructions}

\begin{enumerate}
    \item Connect to your EMG stream (or use Mock for testing)
    \item Click \textbf{Calibrate} while your arm is at rest
    \item Select a channel with strong signal
    \item Contract your forearm muscles to move the bar upward
    \item Try different modes:
    \begin{itemize}
        \item \textbf{Bar}: Simple amplitude indicator
        \item \textbf{Cursor}: Horizontal position control
        \item \textbf{Target Tracking}: Try to match the moving target
    \end{itemize}
    \item Adjust the gain slider to scale the response
\end{enumerate}

\textbf{Question:} How might this concept be extended to control a prosthetic hand with multiple degrees of freedom?

\section{Data Analysis}

Open the Jupyter notebook for guided analysis:

\begin{minted}{bash}
jupyter lab notebooks/Lab1_EMG_Analysis.ipynb
\end{minted}

The notebook covers:
\begin{itemize}
    \item Loading and exploring EMG data
    \item Signal processing pipeline
    \item Frequency domain analysis
    \item Feature extraction
    \item Comparing channels and gestures
\end{itemize}

\section{Deliverables}

Submit the following to MyCourses:

\begin{enumerate}
    \item \textbf{Jupyter Notebook} (completed, with all cells executed)
    \item \textbf{PDF export} of the notebook
    \item \textbf{Lab Report} including:
    \begin{itemize}
        \item Introduction and objectives
        \item Methods (setup, data collection procedure)
        \item Results (plots, feature tables)
        \item Discussion of findings
        \item Answers to embedded questions
    \end{itemize}
\end{enumerate}

\section{Troubleshooting}

\subsection{No Streams Found}
\begin{itemize}
    \item Ensure device is powered on and connected
    \item Check that MyoConnect/BioCapture is running
    \item Try increasing timeout: \texttt{discover\_streams(timeout=5.0)}
\end{itemize}

\subsection{Poor Signal Quality}
\begin{itemize}
    \item Check electrode placement and skin contact
    \item Ensure skin is clean and slightly abraded
    \item Move away from electrical interference sources
    \item Verify ground electrode is properly connected
\end{itemize}

\subsection{Python Errors}
\begin{itemize}
    \item Ensure conda environment is activated
    \item Try reinstalling: \texttt{conda env update -f environment.yml}
    \item Check Python version: \texttt{python --version} (should be 3.11+)
\end{itemize}

\section{Safety Precautions}

\begin{itemize}
    \item \textbf{Never} connect yourself directly to ground while using the BioRadio
    \item \textbf{Never} disassemble the BioRadio or modify its electronics
    \item Keep away from power supplies, oscilloscopes, and other grounded equipment
    \item Remove electrodes gently to avoid skin irritation
    \item Wash hands after handling electrodes
\end{itemize}

\section{References}

\begin{itemize}
    \item Lab Streaming Layer: \url{https://labstreaminglayer.org/}
    \item pylsl Documentation: \url{https://github.com/labstreaminglayer/pylsl}
    \item De Luca, C.J. (1997). The Use of Surface Electromyography in Biomechanics
    \item Merletti, R. \& Parker, P. (2004). Electromyography: Physiology, Engineering, and Non-Invasive Applications
\end{itemize}

\end{document}
